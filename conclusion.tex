%!TEX root = thesis.tex

\chapter{Conclusion}

During the scope of this bachelor project, a custom simulation solution for the robot setup of the IIS Lab was designed, utilizing the V-Rep simulation platform. It was shown, how the simulation was designed and how physical models of the involved robot components were created. The modelling process was described on the example of the Schunk SDH-2 gripper simulation model. Those model components were then used to assemble a simulation scene that reflects the IIS Lab robot setup at a certain configuration, but the scene can easily be adjusted to reflect other settings. \\

The implemented ROS interface corresponds exactly to that of the real robot. The implementation is not tied to a specific simulation scene but to known simulation models. Therefore it is easy to create additional scenes with different settings. The ROS control interface for the simulation models was implemented as a V-Rep simulator plugin. The plugin is loaded on V-Rep startup and identifies known components within the currently active simulation scene. As soon as such a component is identified, it can be controlled by the proper ROS control interface. The plugin design allows to introduce additional components to extend its functionality. \\

The second objective of that project was to provide a motion planning solution that allows to plan and execute complex robot motions without accidental collisions. This was achieved by integrating the motion planning framework MoveIt into the current setup. Therefore we had to create a URDF model that exactly describes the involved robot components and their placement within the environment. Based on that URDF description, the planning framework was set up and configured. The integration process also required the implementation of an additional ROS node that allows the proper execution of time-parametrized trajectories. That node was designed to communicate with the existing ROS control interface and can be configured to interact with the simulator and the real robot as well. Finally, the proper functionality of the planning tools and the simulator was shown on a reference pick and place task that can be executed on both instances.
