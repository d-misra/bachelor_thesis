%!TEX root = thesis.tex

\chapter{Conclusion}

In the course of this bachelor project a custom simulation solution for the robot setup of the IIS Lab has been developed. It utilizes the V-Rep simulation platform. This work explains how the simulation was designed and how physical models of the involved robot components were created. The modelling process was described by example of the Schunk SDH-2 gripper simulation model. The model components have been used to assemble a simulation scene which reflects the IIS Lab robot setup in a certain configuration, but the scene can easily be adjusted for other settings. \\

The implemented ROS interface corresponds exactly to that of the real robot. The implementation is not tied to a specific simulation scene but to known simulation models. Therefore it is easy to create additional scenes with different settings. The ROS control interface for the simulation models was implemented as a V-Rep simulator plugin. The plugin is loaded on V-Rep startup and identifies known components within the currently active simulation scene. As soon as a component is identified, it can be controlled by the proper ROS control interface. The plugin design allows to introduce additional components to extend its functionality. \\

The second objective of the project was to provide a motion planning solution that allows to plan and execute complex robot motions without accidental collisions. This was achieved by integrating the motion planning framework MoveIt into the current setup. Therefore we had to create an URDF model that exactly describes the involved robot components and their placement within the environment. Based on the URDF description, the planning framework was set up and configured. The integration process also required the implementation of an additional ROS node for the proper execution of the time-parametrized trajectories. That node was designed to communicate with the existing ROS control interface and it can be configured to interact with both the simulator and the real robot. Finally, the proper functionality of the planning tools and the simulator was validated with a reference pick and place task that can be executed on both instances.
